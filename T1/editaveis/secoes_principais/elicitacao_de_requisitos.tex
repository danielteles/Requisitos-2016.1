\chapter[Elicitação de requisitos]{Elicitação de requisitos}
A elicitação de requisitos é uma das mais importantes, senão a mais importante atividade quando se fala em desenvolvimento de software. Essa atividade consiste em adquirir o máximo possível de informações e conceitos em relação ao projeto, necessidades dos stakeholders, contexto, restrições, e problema a ser resolvido. Além de ser uma etapa importante para tornar o conhecimento entre o cliente e os desenvolvedores mais homogêneo, de forma que todos tenham o mesmo entendimento do problema a ser resolvido e de que forma ele deverá ser resolvido.
\section{Técnicas para a elicitação de requisitos}
Muitas vezes o cliente não consegue transmitir de forma clara o que deseja, a inexperiência ou falta de preparo do engenheiro de requisitos também é um fator que pode influenciar na hora de elicitar os requisitos. Estes são apenas alguns dos vários problemas que podem surgir durante o processo.
\subsection{Brainstorming}
Permite o surgimento e o refinamento de ideias, é importante que todos participem ativamente e que e encorajem/enriqueçam as ideias dos colegas. Ideias sem relação entre si, quando combinadas podem resultar em soluções criativas para um dado problema. Ao final do brainstorming as ideias que agregam ao produto são revisadas, analisadas e priorizadas.
\subsection{Entrevista}
Tradicional, simples e poderosa, assim pode ser descrita esta técnica. É uma importante técnica na elicitação de requisitos, pois é uma técnica que pode ser aplicada praticamente em qualquer contexto e pode trazer bons resultados no início do processo. Apesar de ser uma técnica simples, é importante que seja feito um planejamento para a entrevista a fim de evitar a perda de foco e também é essencial para que a reunião não se estenda além do necessário.

O contexto é de extrema importância em uma entrevista, uma vez que isso vai possibilitar um maior aproveitamento na dissertação dos assuntos a serem tratados na entrevista. O entrevistador deve estar disposto a escutar e possivelmente mudar de opinião durante a entrevista, porém isso deve ser feito de forma que a produtividade e o foco sejam mantidos.

Ao final da entrevista é importante fazer uma validação de tudo aquilo que foi documentado, verificando assim se a informação adquirida corresponde às necessidades do cliente.
\subsection{Workshop}
É uma técnica de elicitação estruturada, realizada por um grupo composto por stakeholders escolhidos para representar a organização, por um facilitador(geralmente um membro neutro) e pela equipe de analistas. Diferentemente de uma reunião, o workshop tem como objetivo promover o trabalho em equipe e a interação entre os membros. Dentro dessa técnica podem ser utilizadas outras técnicas, como por exemplo o brainstorming, no final são produzidos documentos que contemplam os requisitos e as decisões acordadas a respeito do que será desenvolvido.