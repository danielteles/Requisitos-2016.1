\chapter[Introdução]{Introdução}

Neste documento estão contemplados o planejamento e as discussões que envolvem o gerenciamento dos requisitos de um produto de software. Os requisitos serão elucidados a partir de um processo de Engenharia de Requisitos que será constantemente aprimorado ao longo da disciplina.

Para a realização adequada do projeto, está contido nesse relatório o processo de Engenharia de Requisitos (ER), o qual será criado com base no processo SAFe (Scaled Agile Framework) que será colocado em análise no decorrer deste documento.

As diversas atividades que a equipe de Engenharia de Requisitos irá desenvolver estão explicitadas nesse processo de forma a se alcançar o objetivo final que será produzir um produto de software de alta qualidade e que agregue valor ao cliente. Essas atividades estão de acordo com os cinco pilares a ER que são: elicitação, análise e negociação, documentação, verificação e validação, gestão dos requisitos.

É necessário, para o êxito do projeto, iniciar a partir de uma concepção do trabalho a ser realizado, para isso, a equipe precisa aprofundar-se no conhecimento tanto do funcionamento da empresa cliente como no problema proposto.

  \section{Contexto do cliente}
A empresa júnior de Engenharia Eletrônica da Universidade de Brasília - Faculdade do Gama, a Eletrojun, foi a primeira empresa júnior da FGA. Por ser a empresa mais antiga e atuar em um mercado em expansão no Centro-Oeste, ela é a que tem maior demanda de serviços e o maior número de clientes. Isso gerou uma carência no que diz respeito ao gerenciamento de clientes e seus respectivos projetos.

Para uma melhor compreensão do campo no qual essa empresa e suas necessidades estão concentradas é importante elucidar algumas características inerentes a uma empresa júnior.

\subsection{Movimento Empresa Júnior (MEJ)}
De acordo com a Fundação Getúlio Vargas, “em 1967, alunos da ESSEC – L’École Supérieure des Sciences Economiques et Commerciales, em Paris, sentiram a necessidade de ter conhecimento das ferramentas utilizadas no mercado em que eles trabalhariam num futuro próximo. Assim, foi fundada a Junior ESSEC Conseil, uma associação de estudantes que colocaria em prática os conhecimentos acadêmicos com clientes do mercado”~\cite{fgv}.
A primeira empresa júnior brasileira foi exatamente a Empresa Júnior Fundação Getúlio Vargas, que iniciou um movimento de transformação do cenário acadêmico e empresarial na América Latina. Hoje, o MEJ é uma realidade global constituída por grandes confederações que se somadas, movimentam anualmente mais de 3 milhões de euros.

O MEJ iniciou sua história na Universidade de Brasília em 1993 com a empresa júnior de consultoria empresarial do curso de Administração, AD\&M. Foi a primeira do Centro-Oeste, onde hoje existem mais de 1.200 empresas juniores~\cite{fgv}.

\subsection{A primeira Empresa Júnior da FGA}
Dentro desse contexto, “a Eletronjun - Engenharia Eletrônica Júnior foi criada em 2013, por alunos do curso de Engenharia Eletrônica da Universidade de Brasília, e, desde então, vem aumentando seus horizontes e crescendo como empresa. Com diversos projetos voltados tanto para a universidade quanto para o mercado, a Eletronjun pretende alcançar, cada vez mais, um maior número de pessoas com suas ideias”~\cite{eletronjun}.

O objetivo da empresa é buscar a integração da comunidade do Gama em suas iniciativas, além de propor parcerias com o campo industrial regional, almejando tornar mais acessível o conhecimento a toda a comunidade e desenvolver o meio acadêmico-científico com o auxílio empresarial.
    
A Eletrojun possui uma organização estrutural linear e vertical. No topo da empresa está o presidente, logo abaixo vem os diretores, sendo eles: administrativos, financeiros, de gestão, marketing e projetos. Abaixo dos diretores estão os gerentes, dispostos em 3 áreas principais, que são: administrativa, financeira e de auditoria. E seguindo a hierarquia de cargos estão todos outros membros da instituição. Este tipo de estrutura torna bem claro para cada setor quais atividades devem ser desenvolvidas.

Porém, toda essa organização hierárquica não é suficiente para lidar com a escalabilidade que a empresa busca e, tão logo a empresa emplacara os primeiros empreendimentos, verificou-se a necessidade por uma plataforma que desse suporte para que os projetos, agendas e pessoas fossem integradas de forma a otimizar a produtividade favorecendo a comunicação e organização em diversos aspectos levantados e discutidos ao longo desses três anos de empresa.

Desde o primeiro contato com o cliente até a entrega do produto final, o projeto passará por várias etapas em diferentes setores da empresa. Isso requer um processo minucioso, que se atente a cada característica explícita dos diferentes setores e seus respectivos objetivos a cada iteração.
