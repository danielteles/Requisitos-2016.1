\chapter{Conclusão}

A formulação do processo de Engenharia de Requisitos abordando-se o problema de uma maneira completamente ágil tornou clara a importância do emprego de um sistema flexível que se adapte a projetos cuja mutabilidade é alta. Apesar dos membros da presente equipe já terem trabalhado com métodos ágeis durante a graduação no curso de Engenharia de Software, a curva de aprendizado foi significante quando a prioridade dos estudos se deu no próprio processo de Engenharia de Requisitos.

Os conhecimentos adquidos sobre a eficiência das técnicas da ER foram igualmente engrandecedores. Workshops, entrevistas, \textit{brainstormings} são, a partir dessas experiências vividas, ferramentas essênciais na compreensão do problema e na dinamização do fluxo de conhecimento entre os participantes de um projeto de Engenharia de Software.

O planejamento consistente em um processo foi um ponto chave para a execução com sucesso do projeto. A escolha da metodologia correta para o atual cenário ajudou a equipe a lidar com várias dificuldades durante a execução do cronograma, entre elas, a saída de um membro da equipe já reduzida. A aplicação de diversas técnicas da ER, juntamente com o auxílio de colegas e monitores que sempre estiveram dispostos a colaborar foi providencial para o estado atual do projeto.

\section{Experiência de Execução do Trabalho} \label{expext}

O foco constantemente mantido nas reais necessidades do usuário foram o norte dos trabalhos desenvolvidos na disciplina de Requisitos. Percebeu-se que a real dificuldade da ER é lidar com variáveis não controladas como de costume em outras áreas da Engenharia. De características culturais, sociais e economicas à humor, educação e profissionalismo se aprendeu na prática um pouco. Como captar as reais necessidades dos usuários e transportá-las para um pensamento sistêmico que baseia uma solução em software sem perceber a real dinâmica da essência humana?

Em vários projetos de \textit{software} observam-se falhas decorrentes de problemas na elicitação incorreta de requisitos. Algumas vezes os requisitos são incompletos e não tratam as reais necessiadades do problema, outras vezes podem ser mal elaborados de forma que sejam impreensíveis ou ambíguos.

Existe uma grande dificuldade em compreender o problema, homogeneizar essa compreensão e então identificar corretamente os requisitos de \textit{software}. Para isso foram utilizadas algumas técnicas diferentes mencionadas no decorrer do trabalho, entre elas estão entrevistas, elaboração de diagramas, etc.

Houve uma grande dificuldade nos diálogos com o cliente, que aparentava, em primeira análise, ter compreensão de seu problema e já possuir em mente determinadas características as quais a solução deveria conter. Isso não se verificou nas reuniões posteriores, o que se concretizou na maior dificuldade deste projeto.

\section{Execução da Disciplina}

A dificuldade inicial da equipe foi exatamente compreender o que era Engenharia de Requisitos para então conseguir aplicar o conhecimento científico existente na área na execução do projeto com chances reais de sucesso. Após a fundamentação teórica assimilada, ainda sim, foi necessário um investimento no alinhamento do conhecimento por toda a equipe, de forma que as necessidades internas, assim como as limitações, fossem compreendidas por todos os participantes do grupo.

Após esta fase, iniciou-se uma série de estudos para se estudar as características do cliente e do problema composto. Vários temas foram alvo de análise, como Movimento Empresa Júnior e sua história, necessidades das empresas juniores, o MEJ na UnB e posteriormente no Gama, história e características da Eletronjun, necessidades e objetivos do cliente, etc.

Então, com o respaldo da professora para que a equipe escolhesse o método que mais se aplicasse ao problema proposto, optou-se por uma abordagem completamente ágil de acordo com uma série de estudos sobre as características da equipe, do projeto e do cliente balanceando-as às vantagens e desvantagens de processos ágeis e tradicionais. Essa escolha está descrita fielmente no Trabalho 1.

Como dito na Seção \ref{expext}, a elicitação foi uma grande dificuldade, ainda mais com a saída de um membro de uma equipe já muito pequena. O cliente, com uma visão um tanto inconsistente do \textit{software} enxergava uma solução que para a equipe era impossível de ser construída por erros de concepção em relação a arquitetura do sistema. Para sanar esse impasse, foinecessária a utilização de uma série de técnicas de ER descritas ao longo do trabalho.

As demais fases foram concluídas com êxito e um certo atraso dada a falta de um membro. Entretanto, apesar de tudo, a Engenharia de Requisitos se mostrou uma ótima forma de se atacar um problema e gerenciar os riscos e impactos no decorrer do processo, dando suporte para que conclusão de todas as atividades propostas com sucesso.

A disciplina foi engrandecedora para a formação de um futuro engenheiro de software, estabelecendo parâmetros para que se avalie a qualidade de processos e como eles interferem diretamente no sucesso de um projeto.
