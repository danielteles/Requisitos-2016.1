\chapter[Introdução]{Introdução}

A compreensão do problema está entre as tarefas mais difíceis enfrentadas por um Engenheiro de Software. Afinal de contas, o cliente nem sempre sabe o que é necessário para resolver o seu próprio problema. Projetar e construir um programa de computador elegante que resolva o problema errado não atende às vontades de ninguém \cite{pressman}. 

Diversos estudiosos da Engenharia de Software têm se empenhado em descrever e disseminar a importância da Engenharia de Requitos para ampliar as chances de sucesso de um produto de \textit{software}. Segundo Sommervile (\citeyear{sommerville}) a definição dos requisitos do \textit{software} e sua especificação minuciosa é fundamental para o processo de desenvolvimento obter um \textit{software} com alta qualidade.

Sendo as necessidades de um produto nem sempre compreendidas até pelos próprios clientes, a Engenharia de Requisitos deve atuar para construir e convergir uma ideia de solução que atenda de fato a demanda dos envolvidos. Para isso, de acordo com Thayer e Dorfman (\citeyear{thayer}) é essencial que se preze por determinadas etapas das quais se obtenha um mecanismo apropriado para o entendimento daquilo que o cliente deseja, analisando as necessidades, avaliando a viabilidade, validando a especificação e gerenciando as necessidades à medida que são transformadas em um sistema operacional.

Este documento contém a descrição desse mecanismo através da execução do processo de Engenharia de Requisitos previamente especificado para a construção de um sistema de gestão de clientes para a Empresa Júnior EletronJun do curso Engenharia Eletrônica da Univerdade de Brasília.
