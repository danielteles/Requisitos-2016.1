\chapter[Abordagem da Engenharia de Requisitos]{Abordagem da Engenharia de Requisitos}
“Um conjunto de atividades utilizadas para identificar e comunicar a finalidade de um sistema desoftware, e o contexto no qual será usado. Assim, a ER atua como a ponte entre as necessidades reais dos usuários, clientes, e outros grupos afetados por um sistema de software , e as potencialidades e oportunidades oferecidas pela tecnologia”. (EASTERBROOK, 2004)

“Nos dias atuais, a necessidade de automatização de processos se faz cada vez mais presente. Processos são essenciais dentro de qualquer organização em qualquer área de atuação, inclusive em uma das áreas mais recentes, que a produção de software. A definição de processos de desenvolvimento de software vem com o objetivo de aumentar a produtividade e diminuir os riscos de um projeto”. (VIEIRA, 2003)

“Com o grande crescimento da demanda na produção de software na atualidade os prazos estão cada vez mais curtos e cobrança relacionada a qualidade estão cada vez maiores. A qualidade de um produto de software está diretamente ligada a melhor forma de atender as necessidades do cliente, o que torna um produto totalmente dinâmico. Uma vez que a natureza do produto é dinâmica existem grandes dificuldades no gerenciamento do desenvolvimento, por exemplo: a essência volátil dos requisitos do cliente”. (TAVARES, 2008).

Dados estes fatores, foram criados vários processos dentro da engenharia de software, a fim de amenizar esses problemas, diminuir o tempo gasto e os custos de desenvolvimento. Por exemplo, os processos: cascata, interativo, espiral e incremental.

Cada processo se encaixa melhor em um determinado contexto, geralmente o processo da ER é adaptado de acordo com a equipe, com o cliente e com o projeto. Dado o contexto do trabalho a ser confeccionado nessa disciplina foram propostas duas abordagens, sendo elas: Scaled Agile Framework(SAFe) e o Rational Unified Process(RUP). A seguir daremos uma breve explicação sobre como funciona cada abordagem e será exposta a justificativa da escolha da abordagem e suas atividades.
  \section{SAFe}
O Scaled Agile Framework (SAFe) é um processo que provê uma receita para a aplicação de práticas e princípios ágeis e enxutos em uma escala empresarial. Assim como o Scrum está para uma equipe ágil o SAFe está para um empreendimento ágil. Possui uma abordagem focada na gestão e captura dos requisitos, em vez de no design e implementação.

É composto essencialmente por três níveis: o nível de Time, o nível de Programa e o nível de Portfólio. Para cada nível, existe um nível diferente de narrativa de requisitos: épicos para o Portfólio, features para o Programa e histórias de usuários para o Time. A partir da versão 4.0 deste framework, foi adicionado um novo nível, facultativo, chamado Fluxo de Valor.

O SAFe possui nove princípios fundamentais, baseados nos princípios ágeis, que orientam os papéis e práticas que o tornam efetivo. Esses princípios são:
\begin{itemize}
\item Tenha uma visão econômica
\item Tenha um pensamento sistêmico
\item Assuma variabilidade e preserve opções
\item Construa incrementalmente com ciclos de aprendizagem integrados e rápidos
\item Baseie os marcos em avaliação objetiva dos sistemas 	
\item Descentralize a tomada de decisão
\item Desbloqueie a motivação intrínseca aos profissionais do conhecimento
\item Tenha cadência; sincronize com planejamento cross-domain
\item Visualize e limite o trabalho em progresso, reduza a carga e gerencie o comprimento da fila
\end{itemize}

  \section{RUP}
  \section{Modelo de maturidade}	
Os Modelos de Maturidade de Processos são um referencial usado para:
\begin{itemize}
\item Avaliar a capacidade de processos na realização de seus objetivos;
\item Localizar oportunidades de melhoria de produtividade e qualidade e de redução de custos;
\item Planejar e monitorar as ações de melhoria contínua dos processos empresariais.
\end{itemize}

  \subsection{Modelo CMMI(Capability Maturity Model Integration)}
O CMMI é um modelo de referência que define práticas sejam elas genéricas ou específicas necessárias para o desenvolvimento e avaliação de maturidade no desenvolvimento de softwares em uma organização. As práticas que são abordadas neste modelo são: gerenciamento de requisitos, manipulação de riscos, medição de desempenho, planejamento de trabalho, tomada de decisão, entre outros. O modelo CMMI não pode ser considerado uma metodologia, pois não orienta como deve ser feito, e sim o que deve ser feito. Esse modelo foi desenvolvido pelo SEI(Software Engineering Institute) da Universidade Carnegie Mellon. É uma evolução do CMM, que foi baseado em algumas das ideias mais importantes dos movimentos de qualidade industrial das últimas décadas.
No CMMI uma organização opta por duas representações para a melhoria dos seus processos: Por estágios e contínua.

  \subsection{Modelo MPS-BR (Melhoria de Processo do Software Brasileiro)}
MPS-BR significa Melhoria de Processo do Software Brasileiro, criado pelo Softex e patrocinado pelo MCT. O modelo de maturidade de processos e desenvolvimento de software conhecido como CMMI-DEV foi adaptado para empresas brasileiras, em especial para micro, pequenas e médias empresas, dando origem ao MPS-BR. Essa adaptação foi necessária por que o CMMI-DEV prevê o amadurecimento dos processos em apenas cinco níveis.				
E com o passar do tempo percebeu-se a necessidade de uma funcionalidade mais gradual aqui no Brasil, por isso foi quebrado os cinco níveis do CMMI-DEV em sete, com vemos na figura 2:


Como ilustrado na imagem, o MPS-Br possui possui uma divisão semelhante ao CMMI, entretando o mesmo encontra-se dividido nos níveis constituintes do A ao G, sendo o nível A o mais alto qualitativamente e o G o nível inicial(MPS de Software, São Paulo: SOFTEX, 2012).	
\begin{description}
\item[A.] Em otimização
\item[B.] Gerenciado quantitativamente
\item[C.] Definido
\item[D.] Largamente definido
\item[E.] Parcialmente definido
\item[F.] Gerenciado
\item[G.] Parcialmente gerenciado	
\end{description}			

Os processos referentes a engenharia de requisitos estão descritos no níveis, G e D, correspondentes aos processos de "Gerência de requisitos respectivamente" e "Desenvolvimento de requisitos".

\textbf{Gerência de requisitos}

Como o nome já diz, tem como objetivo gerenciar os requisitos do produto e dos componentes do produto/projeto, além de identificar inconsistências entre requisitos, planos e produtos de trabalho do projeto.

\textbf{Resultados esperados:}
\begin{itemize}
\item \textbf{GRE 1} - O entendimento dos requisitos é obtido junto aos fornecedores de requisitos.
\item \textbf{GRE 2} - Os requisitos são avaliados com base em critérios e objetivos e um comprometimento da equipe técnica com estes requisitos é obtido.
\item \textbf{GRE 3} - A rastreabilidade bidirecional entre os requisitos e os produtos de trabalho é estabelecida e mantida.
\item \textbf{GRE 4} - Revisões em planos e produtos de trabalho do projeto são realizadas visando identificar e corrigir inconsistências em relação aos requisitos.
\item \textbf{GRE 5} - Mudanças nos requisitos são gerenciadas ao longo do projeto.
\end{itemize}

\textbf{Desenvolvimento de requisitos:}
Visa definir os requisitos do cliente, do produto e dos componentes do produto.

\textbf{Resultados esperados:}
\begin{itemize}
\item \textbf{DRE 1} - As necessidades, expectativas e restrições do cliente, tanto do produto quanto de suas interfaces, são identificadas.
\item \textbf{DRE 2} - Um conjunto definido de requisitos do cliente é especificado e priorizado a partir das necessidades, expectativas e restrições identificadas.
\item \textbf{DRE 3} - Um conjunto de requisitos funcionais e não-funcionais, do produto que descrevem a solução do problema a ser resolvido, é definido e mantido a partir dos requisitos do cliente.
\item \textbf{DRE 4} - Os requisitos funcionais e não-funcionais de cada componente do produto são refinados, elaborados e alocados.
\item \textbf{DRE 5} - Interfaces internas e externas do produto e de cada componente do produto são definidas.
\item \textbf{DRE 6} - Conceitos operacionais e cenários são desenvolvidos.
\end{itemize}
  \section{Contexto dentro do projeto}
Neste projeto foi definida a utilização do modelo MPS-Br, levando em conta que ele é bem semelhante ao modelo CMMI e vai possibilitar a produção de resultados parecidos, foram observados outros critérios para a definição do modelo utilizado.

O MPS-Br foi a primeira opção apontada pelo grupo, o fato de ser um modelo brasileiro, voltado para a nossa realidade no que tange o desenvolvimento de software pesou bastante na hora da escolha. O MPS-Br, como dito na sessão 2.3.2, foi pensado para ser utilizado no contexto de micro, pequenas e médias empresas, esse foi outro fator importante na hora da escolha, uma vez que o nosso cliente é uma empresa júnior de engenharia eletrônica. Outro fator que vale a pena destacar é o fato de que o MPS-Br possui uma maior divisão de níveis de maturidade, o que torna o processo para atingir níveis maiores de maturidade algo mais simples, uma vez que por ter mais níveis o número de processos em cada nível é menor do que no CMMI.

  \section{Justificativa}
A metodologia escolhida para a confecção deste trabalho foi uma abordagem ágil, fundamentada no SAFe. Esta escolha foi baseada nas características da equipe e do projeto em si, outro fator importante para a escolha desta metodologia foi o prévio relacionamento que um dos membros do grupo tinha com o nosso cliente, uma vez que o mesmo já havia trabalhado na empresa, além de conhecer a sua estrutura, seus processos e parte da equipe. Isso tornou o processo mais fluido, uma vez que não tivemos a necessidade de fazer uma pesquisa mais profunda a fim de conhecer melhor a empresa.(FINALIZAR/MELHORAR ESSA PARTE, MELHORAR BASTANTE)