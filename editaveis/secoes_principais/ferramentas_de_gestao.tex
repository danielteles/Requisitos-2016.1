\chapter[Ferramentas de gestão de requisitos]{Ferramentas de gestão de requisitos}
O gerenciamento de requisitos é um modelo sistemático para encontrar, documentar, organizar e rastrear os requisitos variáveis de um sistema~\cite{sommerville}. Visando apoiar os responsáveis pelo desenvolvimento de software, foram criadas as ferramentas para gerência de requisitos, essas ferramentas basicamente têm a capacidade de coletar, armazenar, manter e gerenciar mudanças dos requisitos~\cite{tcc_rafael_richa}. Neste capítulo são apresentadas algumas das ferramentas existentes no mercado, estas foram avaliadas e, por fim, foi escolhida a ferramenta que melhor se adequou à gerência de requisitos do projeto.

\section{Análise de Ferramentas}
\subsection{Ferramentas}
As seguintes ferramentas foram analisadas:
\subsubsection*{Jama Software}
\textit{Jama Software} é uma empresa de software com sede em Oregon, Estados Unidos. É a desenvolvedora do \textit{Jama Product Delivery Platform}, um aplicativo baseado em web colaborativo projetado para ajudar as empresas a gerir projetos que se relacionam com o desenvolvimento e introdução de produtos, na figura~\ref{jama} é mostrada a tela inicial desta ferramenta de gestão de projetos, \textit{Jama}.
\begin{figure}[!htbp]
	\centering
	\includegraphics[width=0.8\textwidth]{figuras/jama}
	\caption{Tela inicial da ferramenta de gestão de projetos \textit{Jama}.}
	\label{jama}
\end{figure}

\subsubsection*{RequirementOne}
\textit{RequirementOne} é o nome de uma empresa de desenvolvimento de software com sede na Califórnia, Estados Unidos e também de uma plataforma web de gerenciamento de requisitos, com ela é possível documentar requisitos, realizar trabalhos colaborativos com os \textit{stakeholders}, gerenciar mudanças e identificar problemas, na figura \ref{reqOne} é mostrada a tela incial desta ferramenta.
\begin{figure}[!htbp]
	\centering
	\includegraphics[width=0.8\textwidth]{figuras/reqOne}
	\caption{Tela inicial da plataforma web de gerenciamento de requisitos \textit{RequirementOne}.}
	\label{reqOne}
\end{figure}

\subsubsection*{TargetProcess}
\textit{TargetProcess} é uma aplicação web e mobile de gerenciamento de projetos focada em metodologia de desenvolvimento ágil com suporte para \textit{Scrum} e \textit{Kanban}. O software pode ser personalizado para apoiar abordagens personalizadas de gerenciamento de projetos. Ele está disponível como um SaaS e como um pacote para download. Este software é acessível via web browsers modernos, figura \ref{targetprocess}, e aplicações móveis para dispositivos iPhone, iPad e Android.
\begin{figure}[!htbp]
	\centering
	\includegraphics[width=0.8\textwidth]{figuras/targetprocess}
	\caption{Tela de gerenciamento de épicos da aplicação web \textit{TargetProcess}.}
	\label{targetprocess}
\end{figure}

\subsubsection*{TraceCloud}
\textit{TraceCloud}, fundado e com sede na Califórnia, Estados Unidos, é uma aplicação web desenvolvida com usabilidade, escalabilidade e desempenho em mente. Fornece ferramentas de gerência de projetos, como \textit{TraceMatrix}, \textit{TraceTrees}, análise de impacto de mudanças e etc. Na figura \ref{tracecloud} é possível ver a tela inicial desta ferramenta de gerenciamento.
\begin{figure}[!htbp]
	\centering
	\includegraphics[width=0.8\textwidth]{figuras/tracecloud}
	\caption{Página inicial da aplicação \textit{TraceCloud}.}
	\label{tracecloud}
\end{figure}

\subsection{Critérios de avaliação}
Para realizar uma avaliação sistemática para a escolha da ferramenta a ser utilizada no decorrer do projeto, foram utilizadas sete características como critério de avaliação, os mesmos foram mensuradas em cinco níveis: "Muito ruim", "Ruim", "Aceitável", "Bom" e "Muito Bom". As seguintes características foram avaliadas:
\begin{enumerate}
	\item \textbf{Elicitação:} avaliar armazenamento e gerenciamento de templates e checklists de elicitação e formulários de priorização.
	\item \textbf{Análise:} avaliar poder de análise da ferramenta, como mostrar gráficos, tabelas, resumos e etc.
	\item \textbf{Especificação:} avaliar geração de uma especificação finalizada, incluindo marcas de segurança de páginas, gráficos ou figuras, tabelas definidas e indexadas pelo usuário.
	\item \textbf{Modelagem:} avaliar se a ferramenta fornece notação de modelagem de processo de negócio, modelos de metas, artefatos de SML (\textit{Systems Modeling Language}) e diagramas de fluxo.
	\item \textbf{Verificação e Validação:} avaliar geração de relatórios de requisitos não verificados ou validados e fornecimento de interfaces padrões de acompanhamento de requisitos verificados e validados.
	\item \textbf{Traceabilidade:} analisar a rastreabilidade horizontal e vertical sobre todos os requisitos.
	\item \textbf{Custo:} analisar o custo de utilização da ferramenta, bem como o tempo de avaliação gratuita da mesma. A ferramenta recebe nota "Muito Bom" se e somente se esta for gratuita.
\end{enumerate}

\section{Resultados da análise}

Após analisar as quatro ferramentas pré-selecionadas e citadas acima, levantou-se uma tabela de resultados (tabela \ref{management_tools}).

\begin{table}[!htbp]
\centering
\caption{Comparativo entre as ferramentas de gestão analisadas.}
\label{management_tools}
\begin{tabular}{ccccc}
                                                                                           & \multicolumn{4}{c}{\cellcolor[HTML]{656565}{\color[HTML]{FFFFFF} Ferramentas}}                                                                          \\
\rowcolor[HTML]{656565} 
\cellcolor[HTML]{C0C0C0}Critérios                                                          & {\color[HTML]{FFFFFF} Jama Software} & {\color[HTML]{FFFFFF} RequirementOne} & {\color[HTML]{FFFFFF} TargetProcess} & {\color[HTML]{FFFFFF} TraceCloud} \\
\rowcolor[HTML]{EFEFEF} 
\cellcolor[HTML]{C0C0C0}Elicitação                                                         & Muito Bom                            & Muito Bom                             & Muito Bom                            & Aceitável                         \\
\rowcolor[HTML]{EFEFEF} 
\cellcolor[HTML]{C0C0C0}Análise                                                            & Bom                                  & Muito Bom                             & Muito Bom                            & Bom                               \\
\rowcolor[HTML]{EFEFEF} 
\cellcolor[HTML]{C0C0C0}Especificação                                                      & Bom                                  & Bom                                   & Bom                                  & Bom                               \\
\rowcolor[HTML]{EFEFEF} 
\cellcolor[HTML]{C0C0C0}Modelagem                                                          & Muito Bom                            & Bom                                   & Muito Bom                            & Muito Bom                         \\
\rowcolor[HTML]{EFEFEF} 
\cellcolor[HTML]{C0C0C0}\begin{tabular}[c]{@{}c@{}}Verificação e \\ Validação\end{tabular} & Bom                                  & Muito Bom                             & Muito Bom                            & Bom                               \\
\rowcolor[HTML]{EFEFEF} 
\cellcolor[HTML]{C0C0C0}Traceabilidade                                                     & Aceitável                            & Bom                                   & Muito Bom                            & Bom                               \\
\rowcolor[HTML]{EFEFEF} 
\cellcolor[HTML]{C0C0C0}Custo                                                              & Teste: 30 dias                       & Teste: 14 dias                        & Gratuito \footnote{\label{free_trial_tool}As ferramentas contêm versões pagas, mas também contam com versões gratuitas, essas já atendem ao projeto.}                           & Gratuito \footnoteref{free_trial_tool}                        
\end{tabular}
\end{table}

Como observado na tabela \ref{management_tools}, a ferramenta que melhor se adequa às necessidades do projeto é a \textbf{\textit{TargetProcess}}, além de um design limpo e intuitivo, a ferramenta traz uma série de recursos para se trabalhar com a metodologia ágil, é de fácil rastreabilidade e ainda tem uma versão gratuita para pequenos times, que é o caso do projeto em questão.
